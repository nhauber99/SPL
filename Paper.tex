\documentclass[conference]{IEEEtran}
\usepackage{cite}
\usepackage{amsmath,amssymb,amsfonts}
\usepackage{algorithmic}
\usepackage{graphicx}
\usepackage{textcomp}
\usepackage{xcolor}
\def\BibTeX{{\rm B\kern-.05em{\sc i\kern-.025em b}\kern-.08em
    T\kern-.1667em\lower.7ex\hbox{E}\kern-.125emX}}
\begin{document}

\title{Managing variability in software systems}

\author{\IEEEauthorblockN{Niklas Hauber}
\IEEEauthorblockA{\textit{Johannes Kepler University} \\
Linz, Austria \\
niklas.hauber@solentia.at}
\and
\IEEEauthorblockN{Jonas Reichhardt}
\IEEEauthorblockA{\textit{Johannes Kepler University} \\
Linz, Austria\\
j.reichhardt@tig.at}
}
\maketitle

\begin{abstract}
With growing complexity of software systems, variability of said has become an increasing problem and managing it is thus a crucial part of the present-day software engineering process. It is particularly vital if either many product versions make use of an existing architecture over a long period of time or if a variety of software systems share a common code base. The focus of this paper is to provide an overview of how variation in software systems is caused, how it influences the development process and how it can be dealt with.
\end{abstract}

\section{Introduction}
Building software systems which can adopt to changes from outside as well as within said system represents a vital goal to build software systems that last. When these goals are achieved the effort and therefore costs are reduced for maintenance, but also for further improvements and extension. \\\\
Therefore, a lot of companies design their software such as numerous variants exists, which allows customization to meet the requirements which are laid out by customers and stakeholders. One of the most used practices to achieve a variant-rich software system is software product line engineering (SPLE) which uses concept, processes, modeling techniques for features and function, design patters and respective tools to support the process.[5]\\\\
If one wants to design a system which is prepared for change, the focus should lie on developing components which have loose coupling and strong cohesion. So a module solves one simple task and is as independent as it can be. This leads to the least amount of unwanted side effects in other modules if one module is confronted with unpredicted variability.\\\\
We will try to provide a basic introduction how variability is managed in software systems by collection information from other studies which describe the key aspects of this vast field of research. In the first section we are providing some base knowledge acquired previous studies. After that we will lay out how one can identify the various sources of variability in software systems. After that we will bring up techniques and processes which have the purpose of modeling the variation in a software system. Last but not least we will discuss some of the big tasks which have to be addressed and boil down the described aspects in this paper to a conclussion.
(1/2 page) JR

\section{Background and related work}
In this section the background research which we use to present an overview of the topic.
(1/2 page) NH

\section{Origin of variability}
We introduce why there is variability in the modern software engineering process. So that the reader knows how the problems arrise to have a baseline for presenting solutions for said problems.
(1/2 page) NH

\section{Variability modeling}
The question we want to answer in this section how one can model/describe the changes the software architecture is exposed to and how the dependencies and flexibility can be put into a defined state. 
(1 page) JR

\section{Variability patterns}
Sums up the typical patterns how variability can be introduced.
(1/2 page, maybe merge with section above) NH

\section{???Variability planning???}
Gives an overview of how someone can plan for varability and how to manage it.

\section{Discussion/Challenges}
Discuss why the variability of systems is hard to tackle and why it is not always successful.
(1/4 page) JR

\section{Conclussion}
Make final remarks about the discussed elements.
(1/4 page) NH

\begin{thebibliography}{00}
\bibitem{b1} S. Apel, D. Batory, C. Kästner and G. Saake, "Software Product Lines. In: Feature-Oriented Software Product Lines", Berlin Heidelberg, Springer (2013)
\bibitem{b2} J. van Gurp, J. Bosch and M. Svahnberg, "On the notion of variability in software product lines," Proceedings Working IEEE/IFIP Conference on Software Architecture", Amsterdam, Netherlands, 2001, pp. 45-54
\bibitem{b3} F. Bachmann, L. Bass, "Managing Variability in Software Architectures", Carnegie Mellon University (2001)
\bibitem{b4} M. Galster, D. Weyns, D. Tofan, M. Bartosz, P. Avgeriou, "Variability in Software Systems—A Systematic Literature Review", IEEE Transactions on Software Engineering (2014)
\bibitem{b5} T. Berger, J.-P. Steghöfer, T. Ziadi, J. Robin, and J. Martinez, "The state of adoption and the challenges of systematic variability management in industry", Empirical Software Engineering, vol. 25, pp. 1755-1797, 2020
\end{thebibliography}

\end{document}
